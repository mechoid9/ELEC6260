\documentclass{article}
\usepackage[top=1in, bottom=1in, left=1in, right=1in]{geometry}
% \usepackage{fullpage, fancyhdr}
\usepackage{fullpage}
\usepackage{float}
\usepackage{mathtools}
\usepackage[justification=justified,singlelinecheck=false]{caption}
\usepackage{subcaption}
\usepackage{portland}
\usepackage{graphicx}
\usepackage{paralist}
%\usepackage{setspace}
\setlength{\topmargin}{0.0in}
\setlength{\headheight}{0.5in}
\setlength{\headsep}{0in}
\setlength{\footskip}{9pt}

% General tikz + pgf-umlsd
\usepackage{wrapfig}
\usepackage{listings}
\usepackage{tikz}
\usepackage{pgf-umlsd}
\usepgflibrary{arrows}
\usetikzlibrary{shapes,arrows,shadows}
\usetikzlibrary{shapes.geometric}



% Use so that included code is pretty
\usepackage{listings}
\usepackage{color}

\definecolor{dkgreen}{rgb}{0,0.6,0}
\definecolor{gray}{rgb}{0.5,0.5,0.5}
\definecolor{mauve}{rgb}{0.58,0,0.82}

\lstset{ %
  backgroundcolor=\color{white},  % choose the background color; you must add \usepackage{color} or \usepackage{xcolor}
  basicstyle=\footnotesize,       % the size of the fonts that are used for the code
  breakatwhitespace=false,        % sets if automatic breaks should only happen at whitespace
  breaklines=true,                % sets automatic line breaking
  captionpos=t,                   % sets the caption-position to bottom
  commentstyle=\color{dkgreen},   % comment style
%   deletekeywords={...},           % if you want to delete keywords from the given language
%   escapeinside={\%*}{*)},         % if you want to add LaTeX within your code
%   extendedchar=false,             % lets you use non-ASCII characters; for 8-bits encodings only, does not work with UTF-8
  frame=single,                   % adds a frame around the code
  keywordstyle=\color{blue},      % keyword style
  language={[x86masm]Assembler},                % the language of the code
  morekeywords={LDR,AREA,ENTRY,CODE,DATA,DCD,SPACE,BLT,B,BL.W,ADDS,VMOV,MOVEQ,BEQ,BLE,IMPORT,EXPORT,BGT,},           % if you want to add more keywords to the set
  numbers=left,                   % where to put the line-numbers; possible values are (none, left, right)
  numbersep=5pt,                  % how far the line-numbers are from the code
  numberstyle=\tiny\color{gray},  % the style that is used for the line-numbers
  rulecolor=\color{black},        % if not set, the frame-color may be changed on line-breaks within not-black text (e.g. comments (green here))
  showspaces=false,               % show spaces everywhere adding particular underscores; it overrides 'showstringspaces'
  showstringspaces=false,         % underline spaces within strings only
  showtabs=false,                 % show tabs within strings adding particular underscores
  stepnumber=1,                   % the step between two line-numbers. If it's 1, each line will be numbered
  stringstyle=\color{mauve},      % string literal style
  tabsize=8,                      % sets default tabsize to 2 spaces
  title=\lstname                  % show the filename of files included with \lstinputlisting; also try caption instead of title
}



% \pagestyle{fancyplain}
\pagestyle{myheadings}
\voffset=-0.50in
\topmargin=0.00in 
\headsep=0.25in 
\evensidemargin=0in 
\oddsidemargin=0in 
\textwidth=6.6in 
\textheight=10.0in 

\renewcommand{\topfraction}{0.9}	% max fraction of floats at top
\renewcommand{\bottomfraction}{0.8}	% max fraction of floats at bottom
%   Parameters for TEXT pages (not float pages):
\setcounter{topnumber}{2}
\setcounter{bottomnumber}{2}
\setcounter{totalnumber}{4}     % 2 may work better
\setcounter{dbltopnumber}{2}    % for 2-column pages
\renewcommand{\dbltopfraction}{0.9}	% fit big float above 2-col. text
\renewcommand{\textfraction}{0.07}	% allow minimal text w. figs
%   Parameters for FLOAT pages (not text pages):
\renewcommand{\floatpagefraction}{0.7}	% require fuller float pages
% N.B.: floatpagefraction MUST be less than topfraction !!
\renewcommand{\dblfloatpagefraction}{0.7}	% require fuller float pages
% remember to use [htp] or [htpb] for placement

\title{Assignment \# 8: Problem Set 5}
\date{2/22/2013}
\author{Brian Arnberg}

\markright{Brian Arnberg\hfill ELEC 6260 - Embedded Computing Systems\hfill}     
\setlength{\parindent}{0pt}

\renewcommand{\figurename}{\bf Q3 -}

\begin{document}\label{start}

% \begin{titlepage}
% 	\maketitle
% 	\thispagestyle{empty}
% \end{titlepage}


\section*{Assignment \#9: Problem Set 6 - Due Wed. 02/27/13}
\section{Programming}
While maintaining the same functionality as before, revise the previous program as follows.
\begin{itemize}
	\item Trigger the input handler with an external interrupt signal, generated by a button
           press (rising edge.) Note that you may need to ``debounce'' the switch in software so
           that the program responds only once to any button press.
	\item Change the output handler so that it is executed in response to the SVC (supervisor
           call) instruction.
\end{itemize}
   As before, the program can be tested in RAM or in flash memory, but the final version is to
   be programmed into the flash memory of the microcontroller, so that the program can be
   demonstrated without being connected to the Keil debugger.
   Print and submit the source program, and also email it to me, and bring your programmed
   board to my office to demonstrate the program.



\subsection{Execution}
This assignment was completed by editing the program files from the previous assignment. The files were edited based on example programs offered by \emph{ST Microelectronics} for the ``Discovery'' board. These examples made use of the \texttt{stm32f4\_discovery.h} library, so the program written also made use of this library. Indeed, some of the initialization code (in \texttt{main.c}) was based on what was listed in both the example programs and the libraries that they called.\\

The program worked as expected: Each button press produced a single state change, and each state change occured as expected. 

\subsection{Software Listing}
\lstinputlisting{output_handler.s}
\lstinputlisting{input_handler.s}
\newpage
\lstinputlisting[language=C,frame=single,numbers=left,tabsize=4]{tick_timer.c}
\newpage
\lstinputlisting[language=C,frame=single,numbers=left,tabsize=4]{main.c}
\lstinputlisting[language=C,frame=single,numbers=left,tabsize=4]{MAIN.h}

\newpage
\section{Book Questions}
From the end of Chapter 3, answer questions Q3-24, Q3-25 and Q3-26. These deal with
   cache memory.

% -----------------------------------------------%
% ---------   Q3-24  ----------------------------%      
\setcounter{figure}{23}
\begin{figure}[h]
	\caption{Provide exmples of how each of the following can occur in a typical program:
		}
\begin{description}
	\item[compulsory miss] This can occur at system start-up. At system start-up, no locations have been used, so all memory requests for non-contiguous memory locations will be compulsory misses.
	\item[capacity miss] A capacity miss can occur whenever the CPU needs access to more memory locations than the cache can store. This type of miss will occur whenever the cache is completely occupied. When the CPU needs access to memory not currently in the cache, there will be a capacity miss.  
	\item[conflict miss] A conflict miss can occur when a memory location in the cache is overwritten prematurely. Location A, for instance, is stored in the cache. The CPU then looks for location B, which is not in the cache, and then loads it where location A was stored. Then the CPU looks for location A, which is no longer in the cache, but was where location B now is. A conflict miss has occured. 
\end{description}
\end{figure}

% -----------------------------------------------%
% ---------   Q3-25  ----------------------------%      
\setcounter{figure}{24}
\begin{figure}[h]
\centering
\caption{What is the average memory access time of a machine whose hit rate is 96\%, with a cache access time of 3ns and a main memory access time of 70ns?}
	\begin{tabular}{ l }
		$H_c = .96 \colon T_c = 3ns \colon T_m = 70ns \colon T_a = ?$\\
		\hfill\\
		$T_a = H_c T_c + (1 - H_c)(T_m)$\\
		\hfill\\
		$T_a = .96 \times 3ns + (1-.96)(70ns)$\\
		\hfill\\
		$T_a = 5.68ns$
	\end{tabular}
\end{figure}


% -----------------------------------------------%
% ---------   Q3-26  ----------------------------%      
\setcounter{figure}{25}
\begin{figure}[h]
\centering
\caption{If we want an average memory access time of 6.5ns, our cache access time is 5ns, and our main memory access time is 80ns, what cache hit rate must we achieve?}
	\begin{tabular}{ l }
		$T_a = 6.5ns \colon T_c = 5ns \colon T_m = 80ns \colon H_c = ?$\\
		\hfill\\
		$T_a = H_c T_c + (1 - H_c)(T_m)$\\
		\hfill\\
		$H_c = \frac{T_a - T_m}{T_C - T_m}$\\
		\hfill\\
		$H_c = \frac{6.5ns - 80ns}{5ns - 80ns}$\\
		\hfill\\
		$H_c = .98 = 98\%$
	\end{tabular}
\end{figure}

\label{end}\end{document}


