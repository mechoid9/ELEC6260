\documentclass{article}
\usepackage[top=1in, bottom=1in, left=1in, right=1in]{geometry}
% \usepackage{fullpage, fancyhdr}
\usepackage{fullpage}
\usepackage{float}
\usepackage{mathtools}
\usepackage{caption}
\usepackage{subcaption}
\usepackage{portland}
\usepackage{graphicx}
%\usepackage{setspace}
\setlength{\topmargin}{0.0in}
\setlength{\headheight}{0.5in}
\setlength{\headsep}{0in}
\setlength{\footskip}{9pt}

% Use so that included code is pretty
\usepackage{listings}
\usepackage{color}

\definecolor{dkgreen}{rgb}{0,0.6,0}
\definecolor{gray}{rgb}{0.5,0.5,0.5}
\definecolor{mauve}{rgb}{0.58,0,0.82}

\lstset{ %
  backgroundcolor=\color{white},  % choose the background color; you must add \usepackage{color} or \usepackage{xcolor}
  basicstyle=\footnotesize,       % the size of the fonts that are used for the code
  breakatwhitespace=false,        % sets if automatic breaks should only happen at whitespace
  breaklines=true,                % sets automatic line breaking
  captionpos=t,                   % sets the caption-position to bottom
  commentstyle=\color{dkgreen},   % comment style
%   deletekeywords={...},           % if you want to delete keywords from the given language
%   escapeinside={\%*}{*)},         % if you want to add LaTeX within your code
%   extendedchar=false,             % lets you use non-ASCII characters; for 8-bits encodings only, does not work with UTF-8
  frame=single,                   % adds a frame around the code
  keywordstyle=\color{blue},      % keyword style
  language={[x86masm]Assembler},                % the language of the code
  morekeywords={LDR,AREA,ENTRY,CODE,DATA,DCD,SPACE,BLT,B,BL.W,ADDS,VMOV,MOVEQ,BEQ,BLE,IMPORT,EXPORT,BGT,},           % if you want to add more keywords to the set
  numbers=left,                   % where to put the line-numbers; possible values are (none, left, right)
  numbersep=5pt,                  % how far the line-numbers are from the code
  numberstyle=\tiny\color{gray},  % the style that is used for the line-numbers
  rulecolor=\color{black},        % if not set, the frame-color may be changed on line-breaks within not-black text (e.g. comments (green here))
  showspaces=false,               % show spaces everywhere adding particular underscores; it overrides 'showstringspaces'
  showstringspaces=false,         % underline spaces within strings only
  showtabs=false,                 % show tabs within strings adding particular underscores
  stepnumber=1,                   % the step between two line-numbers. If it's 1, each line will be numbered
  stringstyle=\color{mauve},      % string literal style
  tabsize=8,                      % sets default tabsize to 2 spaces
  title=\lstname                  % show the filename of files included with \lstinputlisting; also try caption instead of title
}



% \pagestyle{fancyplain}
\pagestyle{myheadings}
\voffset=-0.50in
\topmargin=0.00in 
\headsep=0.25in 
\evensidemargin=0in 
\oddsidemargin=0in 
\textwidth=6.6in 
\textheight=10.0in 

\renewcommand{\topfraction}{0.9}	% max fraction of floats at top
\renewcommand{\bottomfraction}{0.8}	% max fraction of floats at bottom
%   Parameters for TEXT pages (not float pages):
\setcounter{topnumber}{2}
\setcounter{bottomnumber}{2}
\setcounter{totalnumber}{4}     % 2 may work better
\setcounter{dbltopnumber}{2}    % for 2-column pages
\renewcommand{\dbltopfraction}{0.9}	% fit big float above 2-col. text
\renewcommand{\textfraction}{0.07}	% allow minimal text w. figs
%   Parameters for FLOAT pages (not text pages):
\renewcommand{\floatpagefraction}{0.7}	% require fuller float pages
% N.B.: floatpagefraction MUST be less than topfraction !!
\renewcommand{\dblfloatpagefraction}{0.7}	% require fuller float pages
% remember to use [htp] or [htpb] for placement

\title{Assignment \# 8: Problem Set 5}
\date{2/22/2013}
\author{Brian Arnberg}

\markright{Brian Arnberg\hfill ELEC 6260 - Embedded Computing Systems\hfill}     
\setlength{\parindent}{0pt}


\begin{document}\label{start}

% \begin{titlepage}
% 	\maketitle
% 	\thispagestyle{empty}
% \end{titlepage}


\section*{Assignment \#8: Problem Set 5 - Due Fri. 02/22/13}

To practice with parallel I/O ports, write a program that creates and displays one of two patterns
on the 4 LEDs on the Discovery board. Under the control of the user push button (the blue
button).

LED3 (orange) = I/O port pin PD13\\
LED4 (green) = I/O port pin PD12\\
LED5 (red)      = I/O port pin PD14\\
LED6 (blue)     = I/O port pin PD15\\
User button (blue) = I/O port pin PA0\\

The program is to operate as follows.
\begin{enumerate}
	\item Initially, all LEDs are off.
	\item On the first press of the user button, the LEDs should be turned on with the following
        pattern:
        LED3 -- LED4 -- LED6 -- LED5 -- ALL OFF (each LED remains ON until ALL OFF)
        This pattern is to be repeated until the next button press. Note that you should see LEDs
        turn on in a counter-clockwise circle.
        Each step of the pattern is to be held for exactly one-half second.
	\item On the next press of the user button, the LED pattern is to change to the following:
        LED3 -- LED5 -- LED6 -- LED4 -- ALL OFF (each LED remains ON until ALL OFF)
        This pattern is to be repeated until the next button press. Note that you should see LEDs
        turn on in a clockwise circle.
        Each step of the pattern is to be held for exactly one-half second.
	\item On the next button press, return to step 1 (all LEDs off). Then repeat steps 1-4
        continuously.
\end{enumerate}


The program is to contain the following modules:
\begin{enumerate}
	\item  An output handler, written in ARM assembly language, which writes patterns to the
       LEDs.
	\item  An input handler, written in ARM assembly language, which tests the user button, and
       sets a global variable.
	\item  A system tick timer interrupt handler, written in C, which is activated every one-half
       second. This routine should call the output handler, if the LEDs are to be changed.
	\item A main program, written in C, which executes in a continuous loop, calling the input
       handle every time through the loop.
   	\item The ``startup code'' for the STM32F4-Discovery board, as found in the Keil installation
       directory: \\ C:/Keil/ARM/Boards/ST/STM32F4-Discovery/Blinky
	\item The STM32F4xx microcontroller's ``include file'', found in the Keil installation directory:\\
       C:/Keil/ARM/INC/ST/STM32F4xx/stm32f4xx.h
\end{enumerate}

\subsection*{Execution}
This assignment was completed by writing two assembly language program files, two C files, and one header file. The project that these files where included in was an edited version of the Keil Blinky project. The contents of the ``Blinky'' directory were copied into a new directory. Then \emph{blink.c} and \emph{leds.c} were removed, and replaced by the five before-mentioned files. The ``startup code''  and ``include file'' were both left unchanged. The program works, though occasionally the ``user'' button seems to detect multiple key-presses when it has only been pressed once, which causes the system to skip one of the above mentioned operation states. 

\newpage
\lstinputlisting{output_handler.s}
\lstinputlisting{input_handler.s}
\lstinputlisting[language=C,frame=single,numbers=left,tabsize=4]{tick_timer.c}
\lstinputlisting[language=C,frame=single,numbers=left,tabsize=4]{main.c}
\lstinputlisting[language=C,frame=single,numbers=left,tabsize=4]{MAIN.h}



%  \begin{figure}[htbp]
%   \centering
%   \includegraphics[width=4.0in,keepaspectratio]{E-Field}
%   \caption{\small{ The E-Field pattern produced by the initial code. }}
%   \label{fig:E-Field}
%   \end{figure}
%  \begin{figure}[htbp]
%   \centering
%   \includegraphics[width=4.0in,keepaspectratio]{Power}
%   \caption{\small{ The normalized power pattern of the system.  }}
%   \label{fig:Power}
%   \end{figure}

\label{end}\end{document}


