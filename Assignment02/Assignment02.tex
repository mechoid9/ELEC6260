\documentclass{article}
\usepackage[top=1in, bottom=1in, left=1in, right=1in]{geometry}
% \usepackage{fullpage, fancyhdr}
\usepackage{fullpage}
\usepackage{float}
\usepackage{mathtools}
\usepackage{caption}
\usepackage{subcaption}
\usepackage{portland}
\usepackage{graphicx}
\usepackage{tikz-uml}
%\usepackage{setspace}
\setlength{\topmargin}{0.0in}
\setlength{\headheight}{0.5in}
\setlength{\headsep}{0in}
\setlength{\footskip}{9pt}


% \pagestyle{fancyplain}
\pagestyle{myheadings}
\voffset=-0.50in
\topmargin=0.00in 
\headsep=0.25in 
\evensidemargin=0in 
\oddsidemargin=0in 
\textwidth=6.6in 
\textheight=10.0in 

\renewcommand{\topfraction}{0.9}	% max fraction of floats at top
\renewcommand{\bottomfraction}{0.8}	% max fraction of floats at bottom
%   Parameters for TEXT pages (not float pages):
\setcounter{topnumber}{2}
\setcounter{bottomnumber}{2}
\setcounter{totalnumber}{4}     % 2 may work better
\setcounter{dbltopnumber}{2}    % for 2-column pages
\renewcommand{\dbltopfraction}{0.9}	% fit big float above 2-col. text
\renewcommand{\textfraction}{0.07}	% allow minimal text w. figs
%   Parameters for FLOAT pages (not text pages):
\renewcommand{\floatpagefraction}{0.7}	% require fuller float pages
% N.B.: floatpagefraction MUST be less than topfraction !!
\renewcommand{\dblfloatpagefraction}{0.7}	% require fuller float pages
% remember to use [htp] or [htpb] for placement

\title{Assignment \# 2: Chapter 1 Questions}
\date{1/22/13}
\author{Brian Arnberg}

\markright{Brian Arnberg\hfill ELEC 6260 - Embedded Computing Systems\hfill Assignment 02\hfill}     
\setlength{\parindent}{0pt}


\begin{document}\label{start}

% \begin{titlepage}
	\maketitle
	\thispagestyle{empty}
% \end{titlepage}


\section*{ Assignment }
From the end of chapter 1, provide answers to the following questions:\\
Q1-2, Q1-4, Q1-11, Q1-15, Q1-16, Q1-17(a), Q1-18(b), Q1-21, Q1-22

% Questions %
\subsection*{ Q1-2  } %Q1-2
Give an example of a requirement on a computer printer.
\\
\emph{Answer:}
An example requirement for a computer printer is that it can print documents produced by a computer. A stricter requirement might be that the computer can print any color document on $8{}^1/_2\times11$'' sized paper. 


\subsection*{ Q1-4  }%Q1-4
Given an example of a specification on a computer printer, giving both type of specification and any required values [sic]. Take your example from an existing product and identify that product.
\\
\emph{Answer:}
An example specification for a computer printer would be its print quality. Print quality is a type of output specification, and can be measured in dots per inch (dpi). The HP Color LaserJet 2600n Printer (Q6455A) can produce black and white documents with a picture quality of up to $600\times600$ dpi. 


\subsection*{ Q1-11 }%Q1-11
Compare and contrast top-down and bottom-up design.
\\
\emph{Answer:}
A top-down design starts from the most abstract description of a system and works towards the most detailed description, while the bottom-up design starts with small components and works towards the bigger system. 

\subsection*{ Q1-15 }%Q1-15
Give a concrete example of how bottom-up information from I/O device hardware design may be useful in refining the architectural design. 
\\
\emph{Answer:}
Consider the design of an embedded thermostat. The thermostat should regulate the temperature of a house. Suppose electronic temperature gauges will be deployed throughout the house, to provide the system with a temperature map, so that it can properly adjust heating and cooling systems on a per-room basis. The electronic temperature gauges will act as I/O devices in the system. The information on their design will affect how they are integrated into the system. For instance, will they return raw or processed data, is there a limit to how many can be in a single system (leading to a maximum number of rooms), how much power does each one draw, etc. With this in mind, if the designers opt to use the LM335A, an analog temperature sensor, they will have to design both a hardware and a software method for translating its analog output to a digital value, and then communicating that digital value to the CPU. The LM335A acts like a Zener diode, so it will have to be the main part of a subsystem, rather than a subsystem itself. The designers could decide to search out a pre-fabricated subsystem, like some Bluetooth thermometer, instead; which would have different design implications. 

\newpage
\subsection*{ Q1-16 }%Q1-16
Create a UML state diagram for the \emph{issue-command()} behavior of the \emph{Controller} class of Figure 1.27.
\\
\emph{Answer:}\\
\begin{tikzpicture}
	\umlbasicstate[name=idle]{Idle State}
	\umlstatedecision[x=5, name=issue]
	\umlbasicstate[x=9, name=have]{Have Command}
	\umlstatedecision[x=9, y=-3, name=type]
	\umlbasicstate[x=6, y=-6, name=speed]{Get Speed}
	\umlbasicstate[x=9, y=-6, name=estop]{estop()}
	\umlbasicstate[x=12, y=-6, name=inert]{Get Inertia}
	\umlstatedecision[x=6, y=-9, name=speed-current]
	\umlstatedecision[x=12, y=-9, name=inert-current]
	\umlbasicstate[x=6, y=-12, name=set-speed]{set-speed()}
	\umlbasicstate[x=14, y=-12, name=set-inert]{set-inertia()}
	\umltrans[anchor1=-15, arg=issue-command(), pos=0.5]{idle}{issue}
	\umlVHVtrans[arm1=-1.5cm, arg=False, pos=1.5]{issue}{idle}
	\umltrans[anchor2=190, arg=True, pos=0.5]{issue}{have}
	\umltrans[arg=cmd-type(), pos=0.5]{have}{type}
	\umlHVtrans[arg=set-command(), pos=0.5]{type}{speed}
	\umltrans[arg=estop(), pos=0.5]{type}{estop}
	\umlHVtrans[arg=set-inertia(), pos=0.5]{type}{inert}
	\umltrans[arg=current-speed(), pos=0.5]{speed}{speed-current}
	\umltrans[arg=same-inertia(), pos=0.5]{inert}{inert-current}
	\umlHVtrans[arg=Yes, pos=0.2]{speed-current}{idle}
	\umltrans[arg=No, pos=0.5]{speed-current}{set-speed}
	\umlVHVtrans[arm1=-2cm, arg=Done, pos=0.2]{set-speed}{idle}
	\umlVHVtrans[arm1=-4.7cm, arg=Yes, pos=0.2]{inert-current}{idle}
	\umlHVtrans[arg=No, pos=0.5]{inert-current}{set-inert}
	\umlVHVtrans[arm1=-2cm, arg=Done, pos=0.2]{set-inert}{idle}
\end{tikzpicture}

\subsection*{ Q1-17(a) }%Q1-17(a)
Show how a \emph{Set-speed} command flows through the refined class structure described in Figure 1.18, moving from a change on the front panel to the required changes on the train:\\
(a) Show it in the form of a collaboration diagram. 
\\
\emph{Answer:}

\subsection*{ Q1-18(b) }%Q1-18(b)
Show how a \emph{Set-inertia} command flows through the refined class structure described in Figure 1.18, moving from a change on the front panel to the required changes on the train:\\
(b) Show it in the form of a sequence diagram.
\\
\emph{Answer:}

\newpage
\subsection*{ Q1-21 }%Q1-21
Draw a state diagram for a behavior that parses the received bits. The machine should check the address, determine the message type, read the parameters, and check the ECC.
\\
\emph{Answer:}
\\
\begin{tikzpicture}
	\umlbasicstate[y=2, name=wait]{Wait}
	\umlNarynode[x=7, name=RM, below]{Recieve Message}
	\umlHVHtrans[arg=sample-message, pos=1.5]{wait}{RM}
	\umlVHVtrans[arm1=4.0cm, arg=False, pos=0.5]{RM}{wait}
	\umlNarynode[x=12, name=ECC, right]{Check ECC}
	\umltrans[arg=True, pos=0.5]{RM}{ECC}
	\umlVHVtrans[arm1=4.0cm, arg=Invalid, pos=0.5]{ECC}{wait}
	\umlNarynode[x=12, y=-3, name=ADD, right]{Check Address}
	\umltrans[arg=Valid, pos=0.2]{ECC}{ADD}
	\umlHVtrans[arg=Invalid, pos=0.1]{ADD}{wait}
	\umlNarynode[x=12, y=-6, name=MSG, right]{Check Message Type}
	\umltrans[arg=Valid, pos=0.2]{ADD}{MSG}
	\umlHVtrans[arg=Invalid, pos=0.1]{MSG}{wait}
	\umlNarynode[x=12, y=-9, name=PAR, right]{Check Parameters}
	\umltrans[arg=Valid, pos=0.2]{MSG}{PAR}
	\umlHVtrans[arg=Invalid, pos=0.1]{PAR}{wait}
	\umlbasicstate[x=12, y=-12, name=issue]{Issue Command}
	\umltrans[arg=Valid, pos=0.2]{PAR}{issue}
	\umlHVtrans[arg=Done, pos=0.1]{issue}{wait}
\end{tikzpicture}


\newpage
\subsection*{ Q1-22 }%Q1-22
Draw a class diagram for the classes required in a basic microwave oven. The system should be able to set the microwave power level between 1 and 9 and time a cooking run up to 59 minutes and 59 seconds in one\--second increments. Include \* classes for the physical interfaces to the telephone line, microphone, speaker, and buttons.  
\\
\emph{Answer:} 
\\
\begin{tikzpicture}
	\umlclass[x=4, name=MO]{Microwave Oven}{}{}
	\umlclass[x=-2, y=-3, name=FP]{Front Panel}{Button: int \\ current-power: int \\ current-time: int \\ cook: bool}{send-command() \\ update-display(): d1, d2, d3, d4}
	\umlclass[x=2, y=-7, name=PD]{Panel Display*}{}{ display-time() \\
		display-power() \\ display-finished() }
	\umlclass[x=-2, y=-7, name=KP]{Keypad*}{}{send-command()}
	\umlclass[x=-2, y=-10, name=BB]{Button*}{button-value}{button-pressed()}
	\umlclass[x=10, y=-3, name=MHC]{Microwave Heater Coil*}{on: bool \\ power: int}{
		set-power() \\ power-on() \\ power-off()}
	\umlclass[x=6, y=-3, name=controller]{Controller}{}{}
	\umluniaggreg[geometry=-|, mult1=1, pos1=0.1, mult2=1, pos2=1.8]{MO}{MHC}
	\umluniaggreg[geometry=-|, mult1=1, pos1=0.1, mult2=1, pos2=1.8]{MO}{FP}
	\umluniaggreg[mult1=1, pos1=0.2, mult2=1, pos2=0.8]{FP}{KP}
	\umluniaggreg[geometry=-|, mult1=1, pos1=0.2, mult2=1, pos2=1.8]{FP}{PD}
	\umluniaggreg[mult1=1, pos1=0.2, mult2=*, pos2=0.8]{KP}{BB}
	\umlVHVuniaggreg[mult1=1, pos1=0.2, mult2=1, pos2=2.8]{MO}{controller}
\end{tikzpicture}

%  \begin{figure}[htbp]
%   \centering
%   \includegraphics[width=4.0in,keepaspectratio]{E-Field}
%   \caption{\small{ The E-Field pattern produced by the initial code. }}
%   \label{fig:E-Field}
%   \end{figure}
%  \begin{figure}[htbp]
%   \centering
%   \includegraphics[width=4.0in,keepaspectratio]{Power}
%   \caption{\small{ The normalized power pattern of the system.  }}
%   \label{fig:Power}
%   \end{figure}

\label{end}\end{document}


