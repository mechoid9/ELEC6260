\documentclass{article}
\usepackage[top=1in, bottom=1in, left=1in, right=1in]{geometry}
% \usepackage{fullpage, fancyhdr}
\usepackage{fullpage}
\usepackage{float}
\usepackage{mathtools}
\usepackage{caption}
\usepackage{sidecap} % enable side captions
\usepackage{subcaption}
\usepackage{portland}
\usepackage{graphicx}
%\usepackage{setspace}
\setlength{\topmargin}{0.0in}
\setlength{\headheight}{0.5in}
\setlength{\headsep}{0in}
\setlength{\footskip}{9pt}

% Use for circuitikz/tikz
\usepackage[american]{circuitikz}
\usepackage{tikz}
\usetikzlibrary{chains}
\usepackage{tikz-dsp}

\DeclareMathAlphabet{\mathpzc}{OT1}{pzc}{m}{it}
\newcommand{\z}{\mathpzc{z}}

% Use so that included code is pretty
\usepackage{listings}
\usepackage{color}

\definecolor{dkgreen}{rgb}{0,0.6,0}
\definecolor{gray}{rgb}{0.5,0.5,0.5}
\definecolor{mauve}{rgb}{0.58,0,0.82}

\lstset{ %
  backgroundcolor=\color{white},  % choose the background color; you must add \usepackage{color} or \usepackage{xcolor}
  basicstyle=\footnotesize,       % the size of the fonts that are used for the code
  breakatwhitespace=false,        % sets if automatic breaks should only happen at whitespace
  breaklines=true,                % sets automatic line breaking
  captionpos=t,                   % sets the caption-position to bottom
  commentstyle=\color{dkgreen},   % comment style
%   deletekeywords={...},           % if you want to delete keywords from the given language
%   escapeinside={\%*}{*)},         % if you want to add LaTeX within your code
%   extendedchar=false,             % lets you use non-ASCII characters; for 8-bits encodings only, does not work with UTF-8
  frame=single,                   % adds a frame around the code
  keywordstyle=\color{blue},      % keyword style
  language={[x86masm]Assembler},                % the language of the code
  morekeywords={LDR,AREA,ENTRY,CODE,DATA,DCD,SPACE,BLT,B,BL.W,ADDS,VMOV,IN,ZAP,MAC,APAC,ADD,SACH,MACD,OUT,},           % if you want to add more keywords to the set
  numbers=left,                   % where to put the line-numbers; possible values are (none, left, right)
  numbersep=5pt,                  % how far the line-numbers are from the code
  numberstyle=\tiny\color{gray},  % the style that is used for the line-numbers
  rulecolor=\color{black},        % if not set, the frame-color may be changed on line-breaks within not-black text (e.g. comments (green here))
  showspaces=false,               % show spaces everywhere adding particular underscores; it overrides 'showstringspaces'
  showstringspaces=false,         % underline spaces within strings only
  showtabs=false,                 % show tabs within strings adding particular underscores
  stepnumber=1,                   % the step between two line-numbers. If it's 1, each line will be numbered
  stringstyle=\color{mauve},      % string literal style
  tabsize=8,                      % sets default tabsize to 2 spaces
  title=\lstname                  % show the filename of files included with \lstinputlisting; also try caption instead of title
}



% \pagestyle{fancyplain}
\pagestyle{myheadings}
\voffset=-0.50in
\topmargin=0.00in 
\headsep=0.25in 
\evensidemargin=0in 
\oddsidemargin=0in 
\textwidth=6.6in 
\textheight=10.0in 

\renewcommand{\topfraction}{0.9}	% max fraction of floats at top
\renewcommand{\bottomfraction}{0.8}	% max fraction of floats at bottom
%   Parameters for TEXT pages (not float pages):
\setcounter{topnumber}{2}
\setcounter{bottomnumber}{2}
\setcounter{totalnumber}{4}     % 2 may work better
\setcounter{dbltopnumber}{2}    % for 2-column pages
\renewcommand{\dbltopfraction}{0.9}	% fit big float above 2-col. text
\renewcommand{\textfraction}{0.07}	% allow minimal text w. figs
%   Parameters for FLOAT pages (not text pages):
\renewcommand{\floatpagefraction}{0.7}	% require fuller float pages
% N.B.: floatpagefraction MUST be less than topfraction !!
\renewcommand{\dblfloatpagefraction}{0.7}	% require fuller float pages
% remember to use [htp] or [htpb] for placement

\title{Assignment \# 6: Problem Set 4, Problem 1}
\date{2/12/2013}
\author{Brian Arnberg}

\markright{Brian Arnberg\hfill ELEC 6260 - Embedded Computing Systems\hfill}     
\setlength{\parindent}{0pt}


\begin{document}\label{start}

% \begin{titlepage}
% 	\maketitle
% 	\thispagestyle{empty}
% \end{titlepage}

\section*{Assignment \# 6: Problem Set 4, Problem 1 - Due 2/13/2013}

\begin{SCfigure}[50][h]
\begin{tikzpicture}


	\matrix[row sep=6.0mm, column sep=08mm]
	{
		%--------------------------------------------------------------------
		\node[dspnodeopen,dsp/label=left] 	(m00) {$x(n)$};   	&
		\node[dspadder]		           	(m01) {};          	&
		\node[dspnodefull,dsp/label=above] 	(m02) {$v_0$};     	&
		\node[dspadder]		           	(m03) {};          	&
		\node[dspnodeopen,dsp/label=right] 	(m04) {$y(n)$};	      	\\
		%--------------------------------------------------------------------
		\node[coordinate]			(m10) {};		&
		\node[coordinate]			(m11) {};		&
		\node[dspsquare]			(m12) {$z^{-1}$};	&
		\node[coordinate]			(m13) {};		&
		\node[coordinate]			(m14) {};		\\
		%--------------------------------------------------------------------
		\node[coordinate]			(m20) {};		&
		\node[dspadder]		           	(m21) {};          	&
		\node[dspnodefull]                 	(m22) {}; 	     	&
		\node[dspadder]		           	(m23) {};          	&
		\node[coordinate]			(m24) {};		\\
		%--------------------------------------------------------------------
		\node[coordinate]			(m30) {};		&
		\node[coordinate]			(m31) {};		&
		\node[dspsquare]			(m32) {$z^{-1}$};	&
		\node[coordinate]			(m33) {};		&
		\node[coordinate]			(m34) {};		\\
		%--------------------------------------------------------------------
		\node[coordinate]			(m40) {};		&
		\node[dspadder]		           	(m41) {};          	&
		\node[dspnodefull]		 	(m42) {};     		&
		\node[dspadder]		           	(m43) {};          	&
		\node[coordinate]			(m44) {};		\\
		%--------------------------------------------------------------------
		\node[coordinate]			(m50) {};		&
		\node[coordinate]			(m51) {};		&
		\node[dspsquare]			(m52) {$z^{-1}$};	&
		\node[coordinate]			(m53) {};		&
		\node[coordinate]			(m54) {};		\\
		%--------------------------------------------------------------------
		\node[coordinate]			(m60) {};		&
		\node[dspadder]		           	(m61) {};          	&
		\node[dspnodefull]                 	(m62) {};     	&
		\node[dspadder]		           	(m63) {};          	&
		\node[coordinate]			(m64) {};		\\
		%--------------------------------------------------------------------
	};

	% Flow accross the top
	\draw[dspconn] (m00) -- (m01);	
	\draw[dspline] (m01) -- (m02);	
	\draw[dspconn] (m02) -- node[near end,above] {$b_0$} (m03);	
	\draw[dspconn] (m03) -- (m04);
	% Go down the middle
	\draw[dspconn] (m02) -- (m12);	
	\draw[dspline] (m12) -- node[very near end,right] {$v_1$} (m22);	
	\draw[dspconn] (m22) -- (m32);	
	\draw[dspline] (m32) -- node[very near end,right] {$v_2$} (m42);	
	\draw[dspconn] (m42) -- (m52);	
	\draw[dspline] (m52) -- node[very near end,right] {$v_3$} (m62);	
	% Branch left
	\draw[dspconn] (m62) -- node[near end,above] {$b_3$} (m63);
	\draw[dspconn] (m42) -- node[near end,above] {$b_2$} (m43);
	\draw[dspconn] (m22) -- node[near end,above] {$b_1$} (m23);
	% Branch left
	\draw[dspconn] (m62) -- node[near end, above] {$-a_3$} (m61);
	\draw[dspconn] (m42) -- node[near end, above] {$-a_2$} (m41);
	\draw[dspconn] (m22) -- node[near end, above] {$-a_1$} (m21);
	% Branch up (on left side)
	\draw[dspconn] (m61) -- (m41);
	\draw[dspconn] (m41) -- (m21);
	\draw[dspconn] (m21) -- (m01);
	% Branch up (on right side)
	\draw[dspconn] (m63) -- (m43);
	\draw[dspconn] (m43) -- (m23);
	\draw[dspconn] (m23) -- (m03);
\end{tikzpicture}
\caption{
Shown to the left is the design of an ``IIR direct form type II filter,'' described in Chapter 5 of the
text. Design a program for the TMS 32C5x DSP, discussed in class, to implement this filter,
using the format of the class examples. Data sample x(n) is to be read from an input port, and
filter output y(n) is to be sent to an output port, as in the ``Low Pass Filter'' example in the class
slides. The values a1, a2, a3, b0, b1, b2, b3 are constants, stored in program memory. Variables
v0, v1, v2, v3 are to be stored in data memory. You may assume that all values are in the ``proper
format'' to use for this calculation, as shown in the examples on the class slides.
	Submit the source program (typed or hand written).
	On the submitted program, show how you have the variables arranged and stored in
         memory.
 }
\end{SCfigure}

\lstinputlisting{PS4_1.s}

% \begin{signalflow}{IIR direct form type II filter}
% 	\begin{scope}
% 		\node[input] (in) 			{$x(n)$};
% 		\node[adder] (a0) [right from=in]	{};
% 		\node[node]  (v0) [right from=a0]	{$v_0$};
% 		\node[adder] (b0) [right from=v0]	{\nodepart{above}{$b_0$}};
% 		\node[output] (yn) [right from=b0]	{$y(n)$};
% 
% 		\path[r>] (in) -- (a0);
% 		\path[r>] (a0) -- (v0);
% 		\path[r>] (v0) -- (b0);
% 		\path[r>] (b0) -- (yn);
% 	\end{scope}
%     % - delay element
% %     \begin{scope}
% %         \node[input]  (in)                   {$x(t)$};
% %         \node[delay]  (del) [right from=in]  {$T$};
% %         \node[output] (out) [right from=del] {$x(t-T)$};
% %         % signal paths
% %         \path[r>] (in)  -- (del);
% %         \path[r>] (del) -- (out);
% %     \end{scope}
%     % - filter
% \end{signalflow}




%  \begin{figure}[htbp]
%   \centering
%   \includegraphics[width=4.0in,keepaspectratio]{E-Field}
%   \caption{\small{ The E-Field pattern produced by the initial code. }}
%   \label{fig:E-Field}
%   \end{figure}
%  \begin{figure}[htbp]
%   \centering
%   \includegraphics[width=4.0in,keepaspectratio]{Power}
%   \caption{\small{ The normalized power pattern of the system.  }}
%   \label{fig:Power}
%   \end{figure}

\label{end}\end{document}


